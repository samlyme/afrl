\documentclass{beamer}

\usetheme{metropolis}
\usepackage{amsmath}
\usepackage{graphicx}

\newcommand{\personcard}[3]{
    \begin{minipage}{10em}
        \centering
        \begin{figure}
            \includegraphics[
                width=8em
            ]{#1}
        \end{figure}
        \textbf{#2}

        \small
        #3
    \end{minipage}%
}


\title{
    Exploring Deep Learning Techniques for Long Term UAV Trajectory Prediction
}
\author{
    Dr. Shokoufeh Mirzaei, Sam Ly, Anna Chiu, Megan Bee, Ray Tan
}
\institute{Cal Poly Pomona}
\date{\today}

\begin{document}
\begin{frame}
    \titlepage
\end{frame}

\begin{frame}
    \frametitle{Introduction}
    \centering

    \begin{columns}
        \begin{column}{10em}
            \personcard{resources/mirzaei.png}{Shokoufeh Mirzaei}{
                Ph.D., IE @ WSU

                Chair and Professor @ CPP IE
            }
        \end{column}
        
        \begin{column}{10em}
            \personcard{resources/sam.png}{Sam Ly}{
                CS Undergrad @ CPP

                Minor in Mathematics

                Minor in Data Science
            }
        \end{column}
    \end{columns}

\end{frame}

\begin{frame}
    \frametitle{Special Thanks}

    \begin{itemize}
        \item \textbf{Mr. Nam Kim} from the CPP IT department for helping us with HPC cluster.
        \item Fellow students Anna Chiu, Megan Bee, Ray Tan, and Sidd Raj for their contributions to the project.
    \end{itemize}
\end{frame}

\begin{frame}
    \frametitle{Table of Contents}

    \begin{enumerate}
        \item Problem definition
        \item Motivation for Machine Learning and Deep Learning
        \item Methodology
        \item Results
        \item Discussion
    \end{enumerate}
\end{frame}

\begin{frame}[standout]
    Problem Definition
\end{frame}

\begin{frame}
    \frametitle{Problem definition}

    We are trying to predict the future positions of a UAV based on historical
    known positional data. Our research centers around making predictions 
    \textbf{0.1-3 seconds} into the future.

    \centering
    \includegraphics[width=0.7\textwidth]{resources/flight_diag.png}
\end{frame}


\begin{frame}
    \frametitle{Problem definition cont.}
    Find a function $F$ that maps a list of \textbf{recorded points} $X$ to a list 
    of \textbf{predicted points} $\hat{Y}$.

    $$ X = \{x_1, x_2, ..., x_n\}, \hat{Y} = \{\hat{y}_1, \hat{y}_2, ..., \hat{y}_m\} $$ 
    $$ x_i \text{, } \hat{y}_j \in R^3$$
    $$ \hat{Y} = F(X) $$
    where $n$ is the length of the input sequence and $m$ is the length of the 
    output/predicted sequence.

    \textbf{Ideally, $\hat{Y}$ should be similar to $Y$.}
\end{frame}

\begin{frame}[standout]
    What is Machine Learning/Deep Learning?
\end{frame}

\begin{frame}
    \frametitle{What is Machine Learning?}

    \centering
    \includegraphics[width=0.9\textwidth]{resources/whymldiag.png}
\end{frame}

\begin{frame}
    \frametitle{What is Machine Learning? cont.}

    \centering
    \includegraphics[width=0.5\textwidth]{resources/aimldl.png}
\end{frame}

\begin{frame}[standout]
    Why Machine Learning?
\end{frame}

\begin{frame}
    \frametitle{Why Machine Learning?}
    
    Predicting the path of a cyclist is a slightly simpler problem.
    
    \centering
    \includegraphics[width=0.9\textwidth]{resources/countersteering.jpg}

    \emph{A slight turn right is actually \textbf{signal} that the bike will
    make a larger turn left!}
\end{frame}

\begin{frame}
    \frametitle{Why Machine Learning? cont.}

    There are \alert{many} maneuvers/patterns used by FPV UAV drones.

    They all have some sort of \alert{setup}, which can be used as a predictive
    signal for its future position.
    
    \centering
    \includegraphics[width=0.8\textwidth]{resources/fpvmaneuvers.png}
\end{frame}

\begin{frame}
    \frametitle{Why Machine Learning? cont.}

    Traditional mathematical techniques break down in non-linear and 
    non-deterministic environments.

    The decisions of the pilot is \textbf{practically impossible} to model 
    mathematically.
\end{frame}

\begin{frame}
    \frametitle{Why Machine Learning? cont.}

    Mathematical models break down in complex systems with noisy data.

    To be feasible, assumptions must be made.

    The dynamics of the UAV may change.

    Machine Learning and \textbf{Deep Learning} algorithms are uniquely good 
    at picking up on these \textbf{signals} in \textbf{noisy} data.
\end{frame}

\begin{frame}[standout]
    How do we use Machine Learning?
\end{frame}

\begin{frame}
    \frametitle{How Machine Learning?}

    $F$ is now \textbf{parameterized} by a set of \textbf{learned weights}
    $\theta$, denoted $F_\theta$.

    The training process continually updates $\theta$, so that the predictions 
    of our model $F_\theta(X) = \hat{Y}$ gets closer to $Y$.

    At each iteration, we find the \textbf{loss} of our model $L(\theta)$.

    Thus, our \alert{optimal} model weights $\theta^*$ is found by minimizing
    the loss function.
    
    $$ \theta^* = \underset{\theta}{\text{arg min}}\  L(\theta)$$

\end{frame}

\begin{frame}
    \frametitle{Existing Techniques}

    Our problem is now a \alert{sequence to sequence} (seq2seq) prediction problem.
    
    Existing techniques:
    \begin{itemize}
        \item {
            \textbf{Encoder-decoder architectures} with various neural networks 
            (RNN, CNN, GRU, etc.)
        }
        \item {
            \textbf{LSTM} models
        }
        \item {
            \textbf{Transformer} models
        }
        \item {
            \textbf{Informer} models
        }
    \end{itemize}

    \emph{These techniques actually form the foundation for LLM's like ChatGPT!}
\end{frame}


\end{document}