\documentclass{article}
\begin{document}

\title{
    Exploring Deep Learning Techniques for Long Term Compute-Constrained UAV Trajectory Prediction
}

\author{
    Shokoufeh Mirzaei
    \and Sam Ly
    \and Anna Chiu
    \and Megan Bee
    \and Ray Tan
    \and Sidd Raj
}

\date{\today}

\maketitle

\begin{abstract}
    The ability to predict the flight paths of UAV in real-time dynamic 
    environments accurately is critical to many applications in commercial and 
    defense environments. UAV flight dynamics are uniquely hard to model due to 
    their non-linear nature, and the numerous unaccountable variables. 
    Furthermore, the influence of the pilot introduces a non-deterministic 
    element to UAV flight paths. Consequently, most  modern trajectory 
    prediction models rely on data-driven techniques to overcome these 
    challenges. In our research, we explore deep learning techniques, taking 
    advantage of an encoder-decoder architecture along with various neural 
    networks for performing sequence to sequence prediction. 
\end{abstract}

\noindent\textbf{Keywords:} Deep Learning, Neural Networks, Drones, Transformers, LSTM, Informer, Encoder-decoder

\section{Introduction}
\subsection{Problem Statement}
The problem of "UAV trajectory prediction" can be formalized as a function $F$ 
that maps a list of length $n$ of \textbf{recorded points} $X$ to a list of 
length $m$ of \textbf{predicted points} $Y$. The values $n$ and $m$ are the 
\textbf{input sequence length} and \textbf{output sequence length} respectively.

Since we are performing this task in 3D space, each \textbf{point} ($x_i$ and 
$y_i$) is actually a vector in $R^3$. Thus,
$$ X = \{x_1, x_2, ..., x_n\}, Y = \{y_1, y_2, ..., y_m\} $$ 
$$ x_i \in R^3, y_j \in R^3 $$
$$ Y = F(X) $$

We can also see that $X$ and $Y$ can also be thought of as an $n \times 3$
matrix and $m \times 3$ matrix respectively. Thus our final definition becomes:

$$ F: R^{n \times 3} \to R^{m \times 3} $$

Viewing $X$ and $Y$ as a list of points is more intuitive, but view them as 
matrices allows us to more clearly visualize computations.

Note that our definition of the input sequence $X$ and output sequence $Y$ does 
not include any information about the time. Instead, the time is encoded
\textbf{implicity} via the order of points $x_i$. Consequently, our model
expects the timestep to remain constant. Other models may use time as a feature
of the input sequence, allowing for predictions accross varying timescales.

\section{Methods}
\section{Results}
\section{Discussion}
\end{document}